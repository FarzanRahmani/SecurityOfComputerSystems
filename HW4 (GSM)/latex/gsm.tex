\documentclass{article}
\usepackage{amsmath,enumitem,fullpage,graphicx,listings,float,sidecap,setspace,xcolor,wrapfig,booktabs,multirow,subcaption,array,minted,hyperref,xepersian,bidi,svg}
%compile with xelatex+shell+escape
\newcolumntype{C}[1]{>{\centering\arraybackslash}m{#1}}

\definecolor{lg}{HTML}{F4F3F3}
\setlength{\fboxsep}{10pt}
\usemintedstyle{borland}
\hypersetup{
    colorlinks=true,
    linkcolor=blue,
    citecolor=green,
    filecolor=magenta,
    urlcolor=cyan
}
\fontsize{14pt}{16pt}\selectfont
\setlatintextfont{IRNazanin.ttf}
\settextfont{IRNazanin.ttf}

\begin{document}
\begin{titlepage}
    \centering
    \begin{figure}[ht]
        \centering
        \includegraphics[width=0.5\textwidth]{iust.png}
    \end{figure}
    \vspace{1cm}
    {\scshape\Huge \textbf{دانشکده مهندسی کامپیوتر} \par}
    \vspace{1cm}
    {\huge\bfseries تمرین امتیازی GSM فصل دوم \par}
    \vspace{1cm}
    {\Large امنیت سیستم‌های کامپیوتری \par}
    \vspace{1cm}
	{\LARGE  مدرس: دکتر ابوالفضل دیانت\par}
    \vspace{1cm}
    {\LARGE  محمدحسین عباسپور، فرزان رحمانی \par}
    \vspace{1cm}
    {\LARGE شماره دانشجویی: ۹۹۵۲۱۴۳۳، ۹۹۵۲۱۲۷1 \par}
    \vspace{1.22cm}
    {\large نیم سال دوم \par}
    {\large سال تحصیلی ۱۴۰۳-۱۴۰۲ \par}
\end{titlepage}
\newpage
\doublespacing
\singlespacing
\newpage
\section{شبکه GSM}
\leavevmode
\\
\setstretch{1.5}
شبکه GSM یک استاندارد بین‌المللی برای ارتباطات سلولی است که برای اولین بار در دهه ۱۹۹۰ معرفی شد. GSM به عنوان یک فناوری نسل دوم (2G) در ارتباطات سلولی شناخته می‌شود و هنوز تا به امروز در بسیاری از کشورها برای ارتباطات سلولی استفاده می‌شود.
\\

\section{ویژگی‌های GSM}
\leavevmode
\\
\setstretch{1.5}
ویژگی‌ها و عملکرد شبکه GSM عبارتند از:

\begin{itemize}
\item 
استفاده از فرکانس‌های تقسیم شده : GSM از تکنیک تقسیم فرکانس برای تقسیم باند فرکانسی استفاده می‌کند. با استفاده از FDMA، فرکانس‌های موجود در یک منطقه جغرافیایی به صورت تقسیم شده بین کاربران تقسیم می‌شوند. این به شبکه GSM امکان ارائه خدمات به چندین کاربر به صورت همزمان را می‌دهد.

\item
 استفاده از TDMA: GSM از TDMA برای تقسیم زمانی فرکانس‌های تقسیم شده برای ارسال اطلاعات استفاده می‌کند. با استفاده از TDMA، زمان ارسال اطلاعات بین کاربران تقسیم می‌شود، به طوری که هر کاربر در یک زمان مشخص قادر به ارسال و دریافت اطلاعات است.


\item
استفاده از SIM : GSM از کارت SIM برای شناسایی و تأیید هویت کاربران استفاده می‌کند. کارت SIM شامل اطلاعات شبکه و کاربر می‌باشد و در دستگاه تلفن همراه قرار می‌گیرد. با استفاده از کارت SIM، کاربران می‌توانند به شبکه GSM متصل شوند و خدمات مخابراتی را دریافت کنند.


\item 
پشتیبانی از خدمات صوتی و داده: GSM امکان ارائه خدمات صوتی (مکالمات) و خدمات داده (از جمله پیامک‌های کوتاه - SMS و ارسال داده‌ها) را فراهم می‌کند. این خدمات به کاربران امکان ارتباط و تبادل اطلاعات را می‌دهند.

\end{itemize}

شبکه GSM به عنوان یک استاندارد بین‌المللی، امکان اتصال و تبادل اطلاعات بین اپراتورهای مختلف را فراهم می‌کند و اجازه می‌دهد تا کاربران در سراسر جهان با هم ارتباط برقرار کنند.
\\

\section{sys Active GSM}
\leavevmode
\\
\setstretch{1.5}
 شبکه GSM را می‌توان با استفاده از دستگاهی به نام sys Active GSM شنود کرد. این دستگاه که در حالت عادی به عنوان Catcher IMSI نیز شناخته می‌شود، یک دستگاه مخصوص است که برای شنود و در برخی موارد تغییر و مداخله در ارتباطات شبکه GSM استفاده می‌شود. این دستگاه قادر است ترافیک بی سیم مربوط به تلفن همراه‌های در حال مکالمه در یک شبکه GSM را شنود کند و حتی می‌تواند خود را در 
میان ارتباط کاربران قرار دهد.
\\

\section{sys Active GSM چگونه کار می‌کند؟}
\leavevmode
\\
\setstretch{1.5}
نحوه کار این دستگاه به صورت کلی عبارت است از:

\begin{itemize}

\item 
شبیه‌سازی یک ایستگاه پایه سلولی: دستگاه Sys Active GSM قادر است به عنوان یک ایستگاه پایه سلولی عمل کند و اطلاعات لازم برای شبیه‌سازی ایستگاه پایه را در اختیار دارد. این ایستگاه پایه سلولی می‌تواند به تلفن‌های همراه در محدوده خود سرویس دهد و به آنها ارتباطی تقلبی ارائه دهد.

\item 
تعامل با تلفن‌های همراه: هنگامی که تلفن همراه‌ها در محدوده تحت پوشش دستگاه Sys Active GSM قرار می‌گیرند، آنها سعی می‌کنند به شبکه ایستگاه پایه متصل شوند. در این مرحله، دستگاه Sys Active GSM به نماینده ایستگاه پایه سلولی می‌نماید و اطلاعات مورد نیاز برای برقراری ارتباط را از تلفن همراه دریافت می‌کند.

\item 
شنود ترافیک ارتباطی: پس از برقراری ارتباط تقلبی با تلفن همراه، دستگاه Sys Active GSM قادر است ترافیک ارتباطی بین تلفن همراه و ایستگاه پایه را شنود کند. این شامل مکالمات صوتی، پیامک‌ها، داده‌ها و سایر ارتباطات است.

\item 
تغییر و مداخله در ارتباطات: علاوه بر شنود، دستگاه Sys Active GSM در برخی موارد می‌تواند به صورت فعال در میان ارتباط قرار گیرد و تغییراتی در ارتباطات اعمال کند. به عنوان مثال، می‌تواند تماس‌ها را قطع کند، پیامک‌ها را مسدود کند یا دستکاری در داده‌های انتقالی انجام دهد.

\end{itemize}

به طور کلی، دستگاه Sys Active GSM با تقلید از ایستگاه پایه سلولی و ایجاد یک ارتباط تقلبی با تلفن همراه‌ها، قادر است ترافیک ارتباطی را شنود کند و در برخی موارد تغییراتی در ارتباطات اعمال کند. با این کار، قادر است به صورت غیرمجاز به اطلاعات حساس کاربران دسترسی پیدا کند.
\\

\section{راه‌های مقابله از Sys Active GSM}
\leavevmode
\\
\setstretch{1.5}


برای جلوگیری از این نوع حملات، سازمان 3GPP (سازمان مشترک تلفن همراه) تلاش کرده است تا استانداردها و روش‌های امنیتی را در شبکه‌های نسل سوم (3G) و نسل چهارم (4G) بهبود بخشد. این تلاش‌ها عمدتاً برای محدود کردن قابلیت ایجاد ارتباطات تقلبی و جعلی و تشخیص و جلوگیری از حملات IMSI Catcher صورت گرفته است. به طور کلی، این تلاش‌ها شامل موارد زیر است:

\begin{itemize}

\item 
استفاده از رمزنگاری: شبکه‌های نسل سوم و چهارم از رمزنگاری قوی تری نسبت به GSM استفاده می‌کنند. این رمزنگاری باعث کاهش امکان شنود و تقلب در ارتباطات می‌شود.


\item
استفاده از الگوریتم‌های امنیتی: استفاده از الگوریتم‌های امنیتی مانند A5/3 در شبکه‌های 3G و الگوریتم‌های امنیتی مبتنی بر AES در شبکه‌های 4G، امکان تقلب و شنود ارتباطات را به شدت کاهش می‌دهند.


\item
تشخیص تقلب: سازمان 3GPP روش‌های تشخیص و جلوگیری از IMSI Catcher را در استانداردها و امکانات شبکه‌های شبکه همراه تعبیه کرده است. این روش‌ها شامل تشخیص تغییرات ناگهانی در پارامترهای شبکه، تشخیص ارتباطات تقلبی و تشخیص تغییرات ناگهانی در مسیرهای ارتباطی هستند. با تشخیص اینگونه حملات، شبکه توانایی جلوگیری از ادامه عملیات تقلبی را دارد.


\item
استفاده از شبکه‌های همراه نسل پنجم (5G): شبکه‌های 5G از تکنولوژی‌ها و استانداردهای امنیتی پیشرفته‌تری نسبت به نسل‌های قبلی استفاده می‌کنند. این تکنولوژی‌ها شامل شناسایی ارتباطات تقلبی، رمزنگاری قوی‌تر، مدیریت دسترسی پیشرفته و امکانات امنیتی دیگر می‌شوند.

\end{itemize}

اگرچه تلاش‌های بسیاری برای جلوگیری از حملات IMSI Catcher انجام شده است، اما همچنان امکان وقوع این نوع حملات وجود دارد. بنابراین، شرکت‌های تولیدکننده شبکه و اپراتورهای تلفن همراه نیز باید بهبودهای امنیتی مستمری را در شبکه‌های خود اعمال کنند تا از حملات احتمالی جلوگیری کنند.

\end{document}
